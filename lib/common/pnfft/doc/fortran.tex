\chapter{Fortran Interface}

\begin{compactitem}
  \item uses Fortran 2003 iso C bindings
  \item all function names are the same as in C interface with one exception: \verb+get_n+ becomes \verb+get_nos+,
        since Fortran is not case sensitive and we need to distinguish between \verb+get_N+ and \verb+get_n+
  \item all C-pointer must be converted to Fortran-pointers but freed as C-pointers
\end{compactitem}


The order of array dimensions flip, since C uses a row-major memory layout, whereas Fortran uses column-major memory order.


2d data decomposition with non-transposed Fourier coefficients
\begin{equation*}
  \hat N_2 \times \hat N_1 / P_1 \times \hat N_0 / P_0
  \nfftarrow
  C_2 \times C_1 / P_1 \times C_0 / P_0
\end{equation*}

2d data decomposition with transposed Fourier coefficients
\begin{equation*}
  \hat N_0 \times \hat N_2 / P_1 \times \hat N_1 / P_0
  \nfftarrow
  C_2 \times C_1 / P_1 \times C_0 / P_0
\end{equation*}

3d data decomposition with non-transposed Fourier coefficients
\begin{equation*}
  \hat N_2 / P_2 \times \hat N_1 / P_1 \times \hat N_0 / P_0
  \nfftarrow
  C_2 / P_2 \times C_1 / P_1 \times C_0 / P_0
\end{equation*}

3d data decomposition with transposed Fourier coefficients with $P_2 = Q_0 Q_1$
\begin{equation*}
  \hat N_0 \times \hat N_2 / (P_1 Q_1) \times \hat N_1 / (P_0 Q_0)
  \nfftarrow
  C_2 / P_2 \times C_1 / P_1 \times C_0 / P_0
\end{equation*}


