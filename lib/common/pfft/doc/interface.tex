%%%%%%%%%%%%%%%%%%%%%%%%%%%%%%%%%%%%%%%%%%%%%%%%%%%%%%%%%%%%%%%%%%%%%%%%%%%%%%%
\chapter{Interface layers of the PFFT library}\label{chap:api}
%%%%%%%%%%%%%%%%%%%%%%%%%%%%%%%%%%%%%%%%%%%%%%%%%%%%%%%%%%%%%%%%%%%%%%%%%%%%%%%

Similar to FFTW the PFFT interface splits into the following two layers
\begin{compactitem}
  \item The basic interface computes a single FFT in parallel.
  \item The advanced interface computes FFTs of multiple arrays, gives a more sophisticated possibility
        to influence the parallel data distribution and adds support of over- and undersampling.
%   \item The guru interface allows to influence the combination of algorithmic steps to yield a parallel FFT.
\end{compactitem}
From basic to advanced the flexibility of the parallel algorithms increases but also the risk to set up the
parallel execution in a wrong way. Users should stick to the basic interface layer as long as it is sufficiently
flexible for their application.

\section{Basic interface layer}



\section{Advanced interface layer}

\chapter{Support}
\href{mailto:michael.pippig@mathematik.tu-chemnitz.de}{michael.pippig@mathematik.tu-chemnitz.de}
